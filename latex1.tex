\documentclass[a4paper,10pt]{article}
\usepackage[italian]{babel}
\usepackage[utf8x]{inputenc}

\title{Documento di \LaTeX}
\author{Giovanni Canarecci}
\date{\today}

\begin{document}
\maketitle

\section{Cominciamo a scrivere qualcosa}
Iniziamo da una citazione interessante:
\begin{quote}
 \item Anche la matematica è una scienza fatta da esseri umani, e perciò ogni tempo, così come ogni popolo, ha un suo spirito.
\end{quote}
\begin{flushright}
 (Hermann Hankel)
\end{flushright}

\section{Appunti per il prossimo mese}
Riportiamo nella tabella 1 gli appuntementi dell'associazione \textit{Croce bianca \& blu}.

\begin{table}[h]
\caption{Elenco eventi}
\begin{tabular}{|c|c|c|l|}
\hline
Mese&Giorno&Città&Manifestazione\\
\hline
\hline
Gennaio&21&Perugia&Tesseramento in Prefettura\\
\hline
Gennaio&28&Terni&Festa del Tesseramento in Prefettura Cena di Carnevale\\
 & & &Cena di Carnevale\\
\hline
Febbraio&15&Gubbio&Ballo dei Bambini\\
\hline
Marzo&18&Terni&I Torneo di Burraco\\
\hline
\end{tabular}
\end{table}

Si ricorda che per partecipare agli appuntamenti sarà necessario:\\
\begin{enumerate}
\item
prenotare non oltre 5 giorni prima dell'evento telefonando a:
\begin{itemize}
\item
Francesca: 0549-2345XX
 \item 
Roberto: 0549-7654XX
\end{itemize}
\item
portare la tessera dell'associazione;
\item
compilare l'apposito \textsc{modulo di registrazione}.

\end{enumerate}

\section{Lettera di presentazione}
Spett.le Ditta\\
via Gramsci 16\\
34563 Torino\\
\begin{flushright}
 18 aprile 2011
\end{flushright}
\textbf{Oggetto:} Consenga documenti
\\
\\
\\
\\
Io sottoscritto, \textit{Edoardo Macchiavelli}, dichiaro di avre ricevuto in data odierna i documenti riguardanti la questione \textsc{arpa}.
\\
\\
\\
\begin{center}
 \textsc{In fede.}\\
\textsc{Dott.Macchiavelli}
\end{center}

\end{document}